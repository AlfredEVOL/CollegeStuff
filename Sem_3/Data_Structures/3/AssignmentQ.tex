% Options for packages loaded elsewhere
\PassOptionsToPackage{unicode}{hyperref}
\PassOptionsToPackage{hyphens}{url}
%
\documentclass[
]{article}
\author{}
\date{}

\usepackage{amsmath,amssymb}
\usepackage{lmodern}
\usepackage{iftex}
\ifPDFTeX
  \usepackage[T1]{fontenc}
  \usepackage[utf8]{inputenc}
  \usepackage{textcomp} % provide euro and other symbols
\else % if luatex or xetex
  \usepackage{unicode-math}
  \defaultfontfeatures{Scale=MatchLowercase}
  \defaultfontfeatures[\rmfamily]{Ligatures=TeX,Scale=1}
\fi
% Use upquote if available, for straight quotes in verbatim environments
\IfFileExists{upquote.sty}{\usepackage{upquote}}{}
\IfFileExists{microtype.sty}{% use microtype if available
  \usepackage[]{microtype}
  \UseMicrotypeSet[protrusion]{basicmath} % disable protrusion for tt fonts
}{}
\makeatletter
\@ifundefined{KOMAClassName}{% if non-KOMA class
  \IfFileExists{parskip.sty}{%
    \usepackage{parskip}
  }{% else
    \setlength{\parindent}{0pt}
    \setlength{\parskip}{6pt plus 2pt minus 1pt}}
}{% if KOMA class
  \KOMAoptions{parskip=half}}
\makeatother
\usepackage{xcolor}
\IfFileExists{xurl.sty}{\usepackage{xurl}}{} % add URL line breaks if available
\IfFileExists{bookmark.sty}{\usepackage{bookmark}}{\usepackage{hyperref}}
\hypersetup{
  hidelinks,
  pdfcreator={LaTeX via pandoc}}
\urlstyle{same} % disable monospaced font for URLs
\usepackage[margin=3cm]{geometry}
\usepackage{color}
\usepackage{fancyvrb}
\newcommand{\VerbBar}{|}
\newcommand{\VERB}{\Verb[commandchars=\\\{\}]}
\DefineVerbatimEnvironment{Highlighting}{Verbatim}{commandchars=\\\{\}}
% Add ',fontsize=\small' for more characters per line
\newenvironment{Shaded}{}{}
\newcommand{\AlertTok}[1]{\textcolor[rgb]{1.00,0.00,0.00}{\textbf{#1}}}
\newcommand{\AnnotationTok}[1]{\textcolor[rgb]{0.38,0.63,0.69}{\textbf{\textit{#1}}}}
\newcommand{\AttributeTok}[1]{\textcolor[rgb]{0.49,0.56,0.16}{#1}}
\newcommand{\BaseNTok}[1]{\textcolor[rgb]{0.25,0.63,0.44}{#1}}
\newcommand{\BuiltInTok}[1]{#1}
\newcommand{\CharTok}[1]{\textcolor[rgb]{0.25,0.44,0.63}{#1}}
\newcommand{\CommentTok}[1]{\textcolor[rgb]{0.38,0.63,0.69}{\textit{#1}}}
\newcommand{\CommentVarTok}[1]{\textcolor[rgb]{0.38,0.63,0.69}{\textbf{\textit{#1}}}}
\newcommand{\ConstantTok}[1]{\textcolor[rgb]{0.53,0.00,0.00}{#1}}
\newcommand{\ControlFlowTok}[1]{\textcolor[rgb]{0.00,0.44,0.13}{\textbf{#1}}}
\newcommand{\DataTypeTok}[1]{\textcolor[rgb]{0.56,0.13,0.00}{#1}}
\newcommand{\DecValTok}[1]{\textcolor[rgb]{0.25,0.63,0.44}{#1}}
\newcommand{\DocumentationTok}[1]{\textcolor[rgb]{0.73,0.13,0.13}{\textit{#1}}}
\newcommand{\ErrorTok}[1]{\textcolor[rgb]{1.00,0.00,0.00}{\textbf{#1}}}
\newcommand{\ExtensionTok}[1]{#1}
\newcommand{\FloatTok}[1]{\textcolor[rgb]{0.25,0.63,0.44}{#1}}
\newcommand{\FunctionTok}[1]{\textcolor[rgb]{0.02,0.16,0.49}{#1}}
\newcommand{\ImportTok}[1]{#1}
\newcommand{\InformationTok}[1]{\textcolor[rgb]{0.38,0.63,0.69}{\textbf{\textit{#1}}}}
\newcommand{\KeywordTok}[1]{\textcolor[rgb]{0.00,0.44,0.13}{\textbf{#1}}}
\newcommand{\NormalTok}[1]{#1}
\newcommand{\OperatorTok}[1]{\textcolor[rgb]{0.40,0.40,0.40}{#1}}
\newcommand{\OtherTok}[1]{\textcolor[rgb]{0.00,0.44,0.13}{#1}}
\newcommand{\PreprocessorTok}[1]{\textcolor[rgb]{0.74,0.48,0.00}{#1}}
\newcommand{\RegionMarkerTok}[1]{#1}
\newcommand{\SpecialCharTok}[1]{\textcolor[rgb]{0.25,0.44,0.63}{#1}}
\newcommand{\SpecialStringTok}[1]{\textcolor[rgb]{0.73,0.40,0.53}{#1}}
\newcommand{\StringTok}[1]{\textcolor[rgb]{0.25,0.44,0.63}{#1}}
\newcommand{\VariableTok}[1]{\textcolor[rgb]{0.10,0.09,0.49}{#1}}
\newcommand{\VerbatimStringTok}[1]{\textcolor[rgb]{0.25,0.44,0.63}{#1}}
\newcommand{\WarningTok}[1]{\textcolor[rgb]{0.38,0.63,0.69}{\textbf{\textit{#1}}}}
\setlength{\emergencystretch}{3em} % prevent overfull lines
\providecommand{\tightlist}{%
  \setlength{\itemsep}{0pt}\setlength{\parskip}{0pt}}
\setcounter{secnumdepth}{-\maxdimen} % remove section numbering
\usepackage{fvextra}
\DefineVerbatimEnvironment{Highlighting}{Verbatim}{breaklines,commandchars=\\\{\}}
\ifLuaTeX
  \usepackage{selnolig}  % disable illegal ligatures
\fi

\begin{document}

\hypertarget{alfred-jophy}{%
\subsubsection{Alfred Jophy}\label{alfred-jophy}}

\hypertarget{cs27}{%
\subsubsection{CS27}\label{cs27}}

\hypertarget{find-crossover-points}{%
\subsection{1. Find Crossover Points}\label{find-crossover-points}}

Consider a list of points on a cartesian plan of the form
\((x,y)\),where\newline \(x_0<x_1<x_2<...<x_n\)\newline
\(y_1<y_1<y_2<...<y_n\)\newline \(x_0 > y_0\)\newline
\(x_n < y_n\)\newline

A crossover point \(i\) in the list is defined where\newline
\(x_i > y_i\)\newline \(x_{i+1} < y_{i+1}\)\newline

Find all crossover points.

\hypertarget{source-code}{%
\subsubsection{Source Code}\label{source-code}}

\begin{Shaded}
\begin{Highlighting}[]
\PreprocessorTok{\#include }\ImportTok{\textless{}stdio.h\textgreater{}}

\KeywordTok{struct}\NormalTok{ CORD}\OperatorTok{\{}
        \DataTypeTok{int}\NormalTok{ x}\OperatorTok{,}\NormalTok{y}\OperatorTok{;}
\OperatorTok{\};}
\KeywordTok{typedef} \KeywordTok{struct}\NormalTok{ CORD CORD}\OperatorTok{;}

\DataTypeTok{void}\NormalTok{ find\_crossover}\OperatorTok{(}\NormalTok{CORD}\OperatorTok{*}\NormalTok{ list }\OperatorTok{,} \DataTypeTok{int}\NormalTok{ length}\OperatorTok{)\{}
        \ControlFlowTok{for}\OperatorTok{(}\DataTypeTok{int}\NormalTok{ i}\OperatorTok{=}\DecValTok{0}\OperatorTok{;}\NormalTok{i}\OperatorTok{\textless{}}\NormalTok{length}\OperatorTok{;}\NormalTok{i}\OperatorTok{++)}
                \ControlFlowTok{if}\OperatorTok{(}\NormalTok{ list}\OperatorTok{[}\NormalTok{i}\OperatorTok{].}\NormalTok{x}\OperatorTok{\textgreater{}}\NormalTok{list}\OperatorTok{[}\NormalTok{i}\OperatorTok{].}\NormalTok{y }\OperatorTok{\&\&}\NormalTok{ list}\OperatorTok{[}\NormalTok{i}\OperatorTok{+}\DecValTok{1}\OperatorTok{].}\NormalTok{x}\OperatorTok{\textless{}}\NormalTok{list}\OperatorTok{[}\NormalTok{i}\OperatorTok{+}\DecValTok{1}\OperatorTok{].}\NormalTok{y }\OperatorTok{)}
\NormalTok{                        printf}\OperatorTok{(}\StringTok{"}\SpecialCharTok{\textbackslash{}n}\StringTok{Crossover at \%d"}\OperatorTok{,}\NormalTok{i}\OperatorTok{);}
\OperatorTok{\}}

\DataTypeTok{int}\NormalTok{ main}\OperatorTok{()\{}
        \DataTypeTok{int}\NormalTok{ length}\OperatorTok{;}
\NormalTok{        CORD list}\OperatorTok{[}\DecValTok{20}\OperatorTok{];}

\NormalTok{        printf}\OperatorTok{(}\StringTok{"Enter the number of points : "}\OperatorTok{);}
\NormalTok{        scanf}\OperatorTok{(}\StringTok{"\%d"}\OperatorTok{,\&}\NormalTok{length}\OperatorTok{);}
\NormalTok{        printf}\OperatorTok{(}\StringTok{"Enter the points : }\SpecialCharTok{\textbackslash{}n}\StringTok{"}\OperatorTok{);}
        \ControlFlowTok{for}\OperatorTok{(}\DataTypeTok{int}\NormalTok{ i}\OperatorTok{=}\DecValTok{0}\OperatorTok{;}\NormalTok{i}\OperatorTok{\textless{}}\NormalTok{length}\OperatorTok{;}\NormalTok{i}\OperatorTok{++)\{}
\NormalTok{                scanf}\OperatorTok{(}\StringTok{"\%d\%d"}\OperatorTok{,\&}\NormalTok{list}\OperatorTok{[}\NormalTok{i}\OperatorTok{].}\NormalTok{x}\OperatorTok{,\&}\NormalTok{list}\OperatorTok{[}\NormalTok{i}\OperatorTok{].}\NormalTok{y}\OperatorTok{);}
\NormalTok{                printf}\OperatorTok{(}\StringTok{"}\SpecialCharTok{\textbackslash{}n}\StringTok{"}\OperatorTok{);}
        \OperatorTok{\}}

\NormalTok{        printf}\OperatorTok{(}\StringTok{"}\SpecialCharTok{\textbackslash{}n}\StringTok{Index : "}\OperatorTok{);}
        \ControlFlowTok{for}\OperatorTok{(}\DataTypeTok{int}\NormalTok{ i}\OperatorTok{=}\DecValTok{0}\OperatorTok{;}\NormalTok{i}\OperatorTok{\textless{}}\NormalTok{length}\OperatorTok{;}\NormalTok{i}\OperatorTok{++)}
\NormalTok{                printf}\OperatorTok{(}\StringTok{"\%d  "}\OperatorTok{,}\NormalTok{i}\OperatorTok{);}
\NormalTok{        printf}\OperatorTok{(}\StringTok{"}\SpecialCharTok{\textbackslash{}n}\StringTok{  X   : "}\OperatorTok{);}
        \ControlFlowTok{for}\OperatorTok{(}\DataTypeTok{int}\NormalTok{ i}\OperatorTok{=}\DecValTok{0}\OperatorTok{;}\NormalTok{i}\OperatorTok{\textless{}}\NormalTok{length}\OperatorTok{;}\NormalTok{i}\OperatorTok{++)}
\NormalTok{                printf}\OperatorTok{(}\StringTok{"\%d  "}\OperatorTok{,}\NormalTok{list}\OperatorTok{[}\NormalTok{i}\OperatorTok{].}\NormalTok{x}\OperatorTok{);}
\NormalTok{        printf}\OperatorTok{(}\StringTok{"}\SpecialCharTok{\textbackslash{}n}\StringTok{  Y   : "}\OperatorTok{);}
        \ControlFlowTok{for}\OperatorTok{(}\DataTypeTok{int}\NormalTok{ i}\OperatorTok{=}\DecValTok{0}\OperatorTok{;}\NormalTok{i}\OperatorTok{\textless{}}\NormalTok{length}\OperatorTok{;}\NormalTok{i}\OperatorTok{++)}
\NormalTok{                printf}\OperatorTok{(}\StringTok{"\%d  "}\OperatorTok{,}\NormalTok{list}\OperatorTok{[}\NormalTok{i}\OperatorTok{].}\NormalTok{y}\OperatorTok{);}

\NormalTok{        find\_crossover}\OperatorTok{(}\NormalTok{list}\OperatorTok{,}\NormalTok{ length}\OperatorTok{);}

        \ControlFlowTok{return} \DecValTok{0}\OperatorTok{;}
\OperatorTok{\}}
\end{Highlighting}
\end{Shaded}

\hypertarget{output}{%
\subsubsection{Output}\label{output}}

\begin{verbatim}
Enter the number of points : 8
Enter the points : 
1
0

2
3

5
4

6
5

7
8

8
10

10
12

12
13


Index : 0  1  2  3  4  5  6  7  
  X   : 1  2  5  6  7  8  10  12  
  Y   : 0  3  4  5  8  10  12  13  
Crossover at 0
Crossover at 3    
\end{verbatim}

\newpage

\end{document}
